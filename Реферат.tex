\abstract{РЕФЕРАТ}

Объем работы равен \formbytotal{lastpage}{страниц}{е}{ам}{ам}. Работа содержит \formbytotal{figurecnt}{иллюстраци}{ю}{и}{й}, \formbytotal{tablecnt}{таблиц}{у}{ы}{}, \arabic{bibcount} библиографических источников и \formbytotal{числоПлакатов}{лист}{}{а}{ов} графического материала. Количество приложений – 2. Графический материал представлен в приложении А. Фрагменты исходного кода представлены в приложении Б.

Перечень ключевых слов: энергопотребление, управление энергоресурсами, программно-информационная система, энергетическая инфраструктура, умные устройства, оптимизация энергопотребления, мониторинг, энергетический менеджмент.

Объектом разработки является программно-информационная система для управления энергопотреблением в зданиях.

Целью выпускной квалификационной работы является создание эффективной системы, позволяющей контролировать и оптимизировать энергопотребление в зданиях, с учетом различных зон и источников энергии.

В ходе разработки системы были выполнены следующие этапы:
\begin{enumerate}
	\item Анализ энергетической инфраструктуры здания, включая источники энергии и оборудование.
	\item Разработка концептуальной модели системы управления энергопотреблением, учитывающей современные подходы к энергетическому менеджменту.
	\item Спроектирована программная система, включая архитектуру, взаимодействие с умными устройствами и пользовательским интерфейсом.
	\item Сконструирована и протестирована программа для мониторинга, управления и оптимизации энергопотребления.
\end{enumerate}

Входными данными для системы являются информация о потреблении энергии различными зонами здания, технические характеристики оборудования, данные о тарифах и стандартах. Выходные данные включают оптимизированный график энергопотребления, отчеты и предупреждения.

Разработанная программно-информационная система предоставляет эффективные инструменты для управления энергопотреблением в зданиях, способствуя оптимизации расходов и повышению энергетической эффективности.

\selectlanguage{english}
\abstract{ABSTRACT}
  
The volume of work is \formbytotal{lastpage}{page}{}{s}{s}. The work contains \formbytotal{figurecnt}{illustration}{}{s}{s}, \formbytotal{tablecnt}{table}{}{s}{s}, \arabic{bibcount} bibliographic sources and \formbytotal{числоПлакатов}{sheet}{}{s}{s} of graphic material. The number of applications is 2. The graphic material is presented in annex A. The layout of the site, including the connection of components, is presented in annex B.

List of keywords: energy consumption, energy resource management, software information system, energy infrastructure, smart devices, energy consumption optimization, monitoring, energy management.

The object of development is a software information system designed for managing energy consumption in buildings.

The objective of the work is to create an effective system capable of monitoring and optimizing energy consumption in buildings, considering various zones and energy sources.

The development stages of the system included:
\begin{enumerate}
	\item Analysis of the energy infrastructure of the building, encompassing energy sources and equipment.
	\item Development of a conceptual model for energy consumption management system, taking into account modern approaches to energy management.
	\item Design of the software system, including architecture, interaction with smart devices, and user interface.
	\item Construction and testing of the program for monitoring, management, and optimization of energy consumption.
\end{enumerate}

Input data for the system include information on energy consumption in different building zones, technical characteristics of equipment, data on tariffs, and standards. Output data consist of an optimized energy consumption schedule, reports, and warnings.

The developed software information system provides efficient tools for managing energy consumption in buildings, contributing to cost optimization and increased energy efficiency.
\selectlanguage{russian}
