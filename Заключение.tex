\section*{ЗАКЛЮЧЕНИЕ}
\addcontentsline{toc}{section}{ЗАКЛЮЧЕНИЕ}

В ходе разработки программно-информационной системы для управления энергопотреблением в зданиях была проведена обширная работа по проектированию и реализации различных компонентов системы. Результатом этой работы стала система, способная эффективно контролировать и оптимизировать энергопотребление, повышая эффективность и уменьшая негативное воздействие на окружающую среду.

Процесс проектирования включал в себя создание детальных диаграмм компонентов, классов и размещения, что позволило внедрить модульную структуру и обеспечить легкость поддержки и расширения системы в будущем. Использование языка программирования Python и фреймворка Django обеспечило высокую гибкость в разработке и поддержке системы.

Реализованный функционал системы включает в себя учет и анализ энергопотребления в зданиях, возможность визуализации данных, а также систему управления, способную реагировать на изменения параметров окружающей среды и внутренних условий здания. Компоненты, такие как классы EnergyConsumption, Company, Room, Electricity, и другие, были разработаны и реализованы для обеспечения полного функционального покрытия поставленных задач.

Модульное тестирование было активно использовано для проверки корректности работы каждого класса и компонента системы. Тесты обеспечивают стабильность и надежность системы в условиях различных сценариев использования.

Разработанная система предоставляет комплексное решение для управления энергопотреблением, что имеет важное значение в современных условиях стремительного развития городов и увеличения экологической ответственности. Полученные результаты положительно влияют на экономию энергоресурсов и содействуют созданию устойчивой и эффективной среды проживания и работы.

Основные результаты работы системы управления энергопотреблением:

\begin{enumerate}
\item Анализ и определение требований: В ходе проекта был проведен глубокий анализ предметной области системы управления энергопотреблением. Выявлены основные компоненты и функциональные возможности, которые необходимо включить в систему.

\item Проектирование и моделирование: Разработана концептуальная модель системы, включающая ключевые классы и их взаимосвязи. Определены требования к системе, учитывающие эффективное управление энергоресурсами, а также адаптацию к географическим и климатическим особенностям.

\item Разработка и тестирование: Осуществлено проектирование и разработка системы управления энергопотреблением. Реализованы ключевые классы, такие как EnergyConsumption, Company, Room, Electricity, с учетом всех функциональных требований. Произведено модульное тестирование каждого класса, что обеспечило высокую степень надежности и исправное взаимодействие компонентов.

\item Интеграция и системное тестирование: Классы успешно интегрированы в единое целое, обеспечивая работоспособность всей системы. Проведено системное тестирование, в ходе которого проверена эффективность управления энергопотреблением в различных сценариях.

\item Опубликованный рабочий проект: Готовый проект представлен в виде работающей системы управления энергопотреблением. Все требования, поставленные перед проектом, успешно реализованы. Система оптимизирует использование энергоресурсов, обеспечивая баланс между коммерческой деятельностью и экологической устойчивостью.

\item Адаптивность и открытый доступ: Разработанный проект обладает адаптивной архитектурой, способной эффективно функционировать в различных условиях. Система находится в открытом доступе, что способствует ее использованию и внедрению в различных областях.

\end{enumerate}

Все эти шаги и достижения подтверждают успешное завершение проекта по созданию системы управления энергопотреблением, предоставляя мощный инструмент для современного и устойчивого управления ресурсами.




