\section*{ВВЕДЕНИЕ}
\addcontentsline{toc}{section}{ВВЕДЕНИЕ}

В современном мире, где энергетические ресурсы становятся все более дефицитными и ценными, эффективное управление энергопотреблением в зданиях приобретает особенное значение. Для достижения устойчивости и снижения негативного воздействия на окружающую среду необходимы инновационные подходы, способные совмещать современные технологии и информационные системы. В этом контексте вступает в игру Программно-информационная система для управления энергопотреблением в зданиях, представляя собой передовое решение, направленное на оптимизацию энергетической эффективности и уменьшение воздействия на климат.

Эта интегрированная система предназначена для комплексного мониторинга, анализа и управления энергопотреблением в зданиях различного назначения, включая офисные здания, торговые центры, промышленные объекты и жилые комплексы. С ее помощью пользователи получают возможность в режиме реального времени отслеживать энергетические показатели, выявлять неэффективные потребители, оптимизировать расход электроэнергии и, таким образом, значительно снижать эксплуатационные расходы.

Основные преимущества программы включают в себя интеллектуальное управление системами освещения, отопления, вентиляции и кондиционирования воздуха, а также возможность удаленного доступа и управления через веб-интерфейс. Это позволяет адаптировать энергопотребление под конкретные потребности, минимизировать потери энергии и создавать оптимальные условия для жизни и работы.

Программно-информационная система для управления энергопотреблением в зданиях не только способствует рациональному использованию ресурсов, но также является важным шагом в направлении создания устойчивого и энергоэффективного общества. Современные технологии в области управления энергопотреблением становятся неотъемлемой частью стратегии устойчивого развития, и наша программа является ключевым инструментом в этом процессе.

Кроме того, Программно-информационная система (ПИС) для управления энергопотреблением в зданиях обеспечивает функционал сбора и анализа данных, что позволяет выявлять тенденции в потреблении энергии и предоставляет ценную информацию для принятия стратегических решений. Алгоритмы помощника встроенного в систему позволяют автоматически оптимизировать процессы управления, учитывая изменяющиеся потребности и внешние факторы.

Безопасность данных является ключевым аспектом работы системы. Программа предусматривает современные методы шифрования и механизмы защиты конфиденциальности, обеспечивая полную безопасность информации о потреблении энергии и других важных параметрах здания.

\emph{Цель данной работы} заключается в создании эффективной программно-информационной системы, способной мониторинга, управления и оптимизации энергопотребления в различных зонах зданий. Для достижения этой цели необходимо решить ряд\emph{ ключевых задач:}
\begin{itemize}
	\item Провести анализ энергетической инфраструктуры зданий с учетом различных источников энергии и зон потребления.
	\item Разработать концептуальную модель программной системы управления энергопотреблением, основанную на передовых методах энергетического менеджмента.
	\item Спроектировать программную систему, включая архитектуру, взаимодействие с умными устройствами и пользовательский интерфейс.
	\item Сконструировать и протестировать программную систему, обеспечивающую эффективное управление и мониторинг энергопотребления.
\end{itemize}

\emph{Структура и объем работы.} Отчет состоит из введения, 4 разделов основной части, заключения, списка использованных источников, 2 приложений. Текст выпускной квалификационной работы равен \formbytotal{page}{страниц}{е}{ам}{ам}.

\emph{Во введении} сформулирована цель работы, поставлены задачи разработки, описана структура работы, приведено краткое содержание каждого из разделов.

\emph{В первом разделе} на стадии описания технической характеристики предметной области приводится сбор информации о деятельности компании, для которой осуществляется разработка сайта.

\emph{Во втором разделе} на стадии технического задания приводятся требования к разрабатываемому сайту.

\emph{В третьем разделе} на стадии технического проектирования представлены проектные решения для web-сайта.

\emph{В четвертом разделе} приводится список классов и их методов, использованных при разработке сайта, производится тестирование разработанного сайта.

В заключении излагаются основные результаты работы, полученные в ходе разработки.

В приложении А представлен графический материал.
В приложении Б представлены фрагменты исходного кода. 
